\section{Comparison of Existing Platforms}\label{sec:comparison-of-existing-platforms}
This section compares two widely used web platforms, WordPress and Squarespace.

\subsection{WordPress}\label{subsec:wordpress}
WordPress is the most similar project to RouteChasm, open-source content management system written in PHP with SQL database.
It is extremely flexible through its plugin and theming ecosystem.
Being open-source enables the developers to modify the core behavior if needed.

The primary advantages of WordPress include:

\begin{itemize}
    \item High flexibility due to open-source access
    \item Large ecosystem of plugins and themes
    \item Strong community support and documentation
\end{itemize}

Excessive plugin usage may introduce inconsistent behavior or even produce security problems.
However, this flexibility comes at the cost of architectural complexity, which is partially addressed by code comments
and in-depth documentation.
The codebase includes many hidden pitfalls and unaddressed technical debts.

From a developer perspective, WordPress is powerful but complex architecture accumulated over the years of development.
RouteChasm is made for developers first and users second.
If the developers are satisfied with the tooling then they are going to make amazing experience for the user.

\subsection{Squarespace}\label{subsec:squarespace}
Squarespace is a proprietary, fully hosted website builder focused on ease of use and visual design.
Even though Squarespace is closed-source system, it still provides viable plugin system.

It may seem like closed-systems are more secure over the open-source projects but the opposite is true.
When security flaw is found in open-source system it is either fixed before it has a chance to be exploited
or the issue is fixed before it can negatively impact large number of users.
For this reason RouteChasm is open-source project.

The main advantages of Squarespace include:

\begin{itemize}
    \item Extremely fast setup and deployment
    \item Integrated hosting and maintenance
    \item Consistent user experience and stability
    \item Minimal technical knowledge required
\end{itemize}

The primary limitation of Squarespace is restricted extensibility of the core systems.
Developers cannot access the underlying backend architecture to host the websites on their own servers or to modify
core behavior.

Squarespace policy places users over developers, and thus it is great for less tech-savvy users, but not so good for
team of developers.

\subsection{Comparison Summary}\label{subsec:comparison-summary}
WordPress emphasizes openness and extensibility but suffers from architectural complexity and technical debts.
Squarespace prioritizes simplicity but lacks customization needed for developers.

From user experience I find Squarespace website editor easier to use than what WordPress provides.

Neither platform fully satisfies the requirements of a system that aims to provide:

\begin{itemize}
    \item Strong foundational architecture and modern code base
    \item Full control over backend behavior
    \item Extensibility without dependency-heavy ecosystems
\end{itemize}

RouteChasm positions itself between these two approaches:
maintaining developer control and being open-source project similar to WordPress
while providing easy to use editor like the one Squarespace uses.
