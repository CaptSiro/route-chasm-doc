\section{Requirements}\label{sec:requirements}
This section defines the functional and non-functional requirements of RouteChasm.

\subsection{Functional Requirements}\label{subsec:requirements-functional_requirements}
These requirements specify how the system implements AI tooling, core systems, and expected behavior from developers or users.

\subsubsection{Framework Requirements}
\begin{itemize}
    \item FR1: The framework must provide a deterministic routing system.
    \item FR2: Routing system must support common HTTP methods.
    \item FR3: Routing system must be modular, such that part of the routing tree can be moved to different path
        and all the bindings work as expected.
    \item FR4: The framework has the ability to automatically generate CRUD operations for database tables.
    \item FR5: Models must map database columns to properties without extra boilerplate.
    \item FR6: The framework must provide form-handling subsystem.
    \item FR7: The system has the ability to automatically generate forms based on model structure.
    \item FR9: The framework must provide a component-based editor capable of creating and maintaining webpages from reusable components.
    \item FR10: Editor components must be rendered on client from stored configuration.
\end{itemize}

\subsubsection{AI Generation Requirements}
\begin{itemize}
    \item AIR1: The system must include an AI-powered translation unit capable of translating static text and text templates for supported languages.
    \item AIR3: The framework must support generating new webpage components.
    \item AIR4: Users must be able to request static webpage templates that are created by the AI system.
    \item AIR5: The AI generation process must follow the project’s naming conventions and structural constraints.
\end{itemize}

\subsection{Non-Functional Requirements}\label{subsec:requirements-non_functional_requirements}

\begin{itemize}
    \item NFR1: The system must prioritize ease of development, readability, and extensibility.
    \item NFR2: New modules, components, and templates must be easy to add and maintain.
    \item NFR3: The framework must perform efficiently enough for small-to-medium projects.
    \item NFR4: All core features must be implemented without relying on third-party libraries to ensure full control and consistent implementation.
    \item NFR5: Code must follow the conventions defined by the framework’s philosophy.
    \item NFR6: The framework systems must follow a unified design approach.
    \item NFR8: Routing and form generation must be deterministic.
\end{itemize}
